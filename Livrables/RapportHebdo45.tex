\documentclass{article}
\usepackage[utf8]{inputenc}
\usepackage[T1]{fontenc}
\title{Rapport hebdomadaire}
\author{Matéo Rémi \& Plasse Vincent}
\date{11/11/2012}
\begin{document}
\maketitle

\section{Actions effectuées}
Préparation de la soutenance (3h) Matéo - Plasse \\
Soutenance (40min) Matéo - Plasse \\
Amélioration du rapport (4h) Matéo - Plasse  \\


\section{À faire}
Rédiger les parties de modélisation et de spécification \\
Faire une réunion d'avancement avec Nicolas Normand et si possible Eric Remy\\

\section{Notes}
La soutenance a été pour nous l'occasion de faire le point sur les différentes lacunes qui étaient présentes dans notre rapport afin que nous puissions le rendre plus complet. En voici une liste que nous espérons être exaustive :
\begin{itemize}
	\item Présenter le contexte dans lequel s'inscrit le projet, suite à la réunion DGtal du 7 Juin 2012 et notament les rôles qu'ont Eric Remy et Nicolas Normand suite à cette réunion
	\item Préciser le contexte dans lequel les évolutions que nous allons apporter vont se situer (2-D, 3-D, n-D, 6-connexité, etc.)
	\item Ajout d'une référence bibliographique pour la présentation de la connexité
	\item expliquer pourquoi le rapport est en anglais
	\item Ajouter des introductions/conclusions dans les chapitres qui n'en ont pas
	\item Préciser dans le rapport qu'il s'agit pour l'instant d'un contenu en cours de rédaction et qu'il n'est pas terminé
	\item Rajouter les ressources allouées aux tâches dans le Gantt
\end{itemize}
Ces modifications ont été effectuées et la nouvelle version du rapport sera envoyé le lundi 12 novembre suite à la réunion d'avancement.

\end{document}
