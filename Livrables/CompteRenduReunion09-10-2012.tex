\documentclass{article}
\usepackage[utf8]{inputenc}
\usepackage[T1]{fontenc}
\title{Compte Rendu de Reunion}
\author{Matéo Rémi \& Plasse Vincent}
\date{09/102012}
\begin{document}
\maketitle

\section{Points abordées}
M. NORMAND a déjà fait un pull request avec DGtal pour son algorithme de transformation en distance
Problème : DGtal ne dispose pas de fonctions élémentaires (chanfrein, distance simplifiée, etc)

Modèle habituel, utilisé par M. NORMAND dans son pull request : filtre d'images ligne par litre avec un à n filtres consommateurs et en sortie un producteur d'image
Consommateur : beginOfImage()
	processPixels()
	endOfImage()
Problème : non compatible avec DGtal qui lit l'image d'un coup : i/o indépendant de DGtal, que en 2D

Du coup : mieux essayer de mettre les algos classiques, les utiliser de façon plus générale
Distance de chanfrein
Distance de voisinages

Cartes de distance :
Distance dans un Ensemble E : fonction de E dans E d : ExE -> R+
d(x,y) >= 0
d(x,y) = 0 <=> x=y
d(x,y) = d(y,x)
Plus en français d(x,y) <= d(x,z)+d(z,y) : métrique

Classe de distance : path-based
distance = longueur du plus court chemin
du coup
qu'est ce qu'un chemin ? Comment en définir sa longueur ?
Permet de définir une distance

Rosenfeld & Pfalk : 1966 & 1968
4-voisinage et 8-voisinage

Chemin connexe : ensemble de points connexes deux à deux

Carte des distances DTx(p) = min {d(q,p) : q non inclu ds X} distance du fond vers le point
DTx(p) = max{r : B<(O,r) inclu dans X} : rayon du plus grand disque de centre p et contenu dans la forme X

d4 : DTx(p) = 0 si pas dans l'image, 1 + min DTx(p-v), v appartenant à N4 (4 voisinage), v vecteur

alorithme séquentiel : on divise en deux balayages : 1er : chemin qui proviennent de haut et gauche, 2nd : bas et droite
Pour 8-voisinage :
1er : gauche, haut-gauche, haut et haut-droite, 
2nd : droite et 3 bas

Pondération : voir chanfrein

Fonctionne car une distance ne change pas si l'ordre du chemin change. Un seul chemin minimal suffit

Chemin à séquence de voisinages : Séquence B avec utilisation de types de voisinage avec un ordre spécifique
Ex : distance octogonale (4co, 8co, 4co, 8co, etc)
Pb : l'algo ne marche plus dans ce contexte

      
\end{document}
