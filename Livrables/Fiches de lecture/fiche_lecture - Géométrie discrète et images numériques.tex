\documentclass[a4paper,11pt]{report}

\usepackage[pdftex, 	 
bookmarks = true, 	% Signets
bookmarksnumbered = true, 	% Signets numérotés
pdfpagemode = None, 	% Signets/vignettes fermé à l'ouverture
pdfstartview = FitH, 	% La page prend toute la largeur
pdfpagelayout = SinglePage, 	% Vue par page
colorlinks = true, 	% Liens en couleur
urlcolor = magenta, 	% Couleur des liens externes
pdfborder = {0 0 0} 	% Style de bordure : ici, pas de bordure
]{hyperref} 	% Utilisation de HyperTeX
\usepackage{verbatim} 
\usepackage[utf8]{inputenc}
\usepackage[T1]{fontenc}
\usepackage[francais]{babel}
\usepackage{epstopdf}
\usepackage{epsfig}
\hypersetup{
	colorlinks,
	citecolor=red,
	linkcolor=black,
	urlcolor=blue}


\begin{document}

\section{\emph{Géométrie discrète et images numériques}}


\section{Résumé}

The purpose of this book is to present a state of the art of the discrete geometry, from its origines to its objectifs. It also defines differents parts of this science such as :
 - The basements
 - Arithmetic tools
 - Topology and surfaces
 - combinatory maps
 - discrete distances
 - discrete layouts
 - affine transformations
 - topology transformations
 - median axis representation
 - projection-based transformations and reconstructions
 - ...

\section{Analyse}

This book is a great base for us to understand what discrete geometry is. It is also a good start for our personal state of the art since its presents different notions related to our subject such as discrete distances, median axis representation and projection-based transformations and reconstructions. Since it's a very recent book written by tens of researchers, its informations are very accurate.


@book{geometriediscreteetimagesnumeriques,
title={Géométrie discrète et images numériques},
author={Francis Castanié, Henri Maître},
  year={2007},
  publisher={Hermès sciences Lavoisier},
}

\end{document}
