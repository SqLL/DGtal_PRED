\documentclass[a4paper,11pt]{report}

\usepackage[pdftex, 	 
bookmarks = true, 	% Signets
bookmarksnumbered = true, 	% Signets numérotés
pdfpagemode = None, 	% Signets/vignettes fermé à l'ouverture
pdfstartview = FitH, 	% La page prend toute la largeur
pdfpagelayout = SinglePage, 	% Vue par page
colorlinks = true, 	% Liens en couleur
urlcolor = magenta, 	% Couleur des liens externes
pdfborder = {0 0 0} 	% Style de bordure : ici, pas de bordure
]{hyperref} 	% Utilisation de HyperTeX
\usepackage{verbatim} 
\usepackage[utf8]{inputenc}
\usepackage[T1]{fontenc}
\usepackage[francais]{babel}
\usepackage{epstopdf}
\usepackage{epsfig}
\hypersetup{
	colorlinks,
	citecolor=red,
	linkcolor=black,
	urlcolor=blue}


\begin{document}

\section{\emph{<Le titre d'un article>}}

Pour chaque article et ouvrage de la bibliographie (éventuellement pour un site \emph{web} \emph{complet} et \emph{pérenne}), il faut établir une fiche de lecture -- succincte -- comportant (i)~un résumé << objectif >> et (ii)~une analyse.

\section{Résumé}

Le résumé doit faire apparaître l'idée phare de l'article (ou les idées principales d'un ouvrage plus important -- il peut étre étudié selon ses différents chapitres). Il faut rapporter les analyses, expérimentations et conclusions établies par les auteurs.

\section{Analyse}

C'est dans cette seconde partie que l'analyse de l'article a lieu. Il s'agit de vérifier l'exactitude du document (des erreurs sont toujours possibles, la littérature nous l'apprend !), et d'établir le lien avec le sujet de ce projet.


@book{introductionboost,
title={Beyond the C++ Standard Library an Introduction to Boost},
author={Björn Karlsson},
  year={2006},
  publisher={Upper Saddle River (N.J.) : Addison-Wesley, cop. 2006},
	url={http://www.sudoc.fr/093832540}
}

\end{document}
