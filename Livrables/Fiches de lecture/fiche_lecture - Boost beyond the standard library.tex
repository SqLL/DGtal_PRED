
\documentclass[a4paper,11pt]{report}

\usepackage[pdftex, 	 
bookmarks = true, 	% Signets
bookmarksnumbered = true, 	% Signets numérotés
pdfpagemode = None, 	% Signets/vignettes fermé à l'ouverture
pdfstartview = FitH, 	% La page prend toute la largeur
pdfpagelayout = SinglePage, 	% Vue par page
colorlinks = true, 	% Liens en couleur
urlcolor = magenta, 	% Couleur des liens externes
pdfborder = {0 0 0} 	% Style de bordure : ici, pas de bordure
]{hyperref} 	% Utilisation de HyperTeX
\usepackage{verbatim} 
\usepackage[utf8]{inputenc}
\usepackage[T1]{fontenc}
\usepackage[francais]{babel}
\usepackage{epstopdf}
\usepackage{epsfig}
\hypersetup{
	colorlinks,
	citecolor=red,
	linkcolor=black,
	urlcolor=blue}

\begin{document}


\section{\emph{Beyond the standard library an introduction to boost}}

The book is an introductions to boost that means that he will help you
to discover the library boost. He will introducing the different types
and functions of boost. The author
Björn Karlsson is a senior software Engineer at ReadSoft.

\section{Résumé}

In this book Bjorn Karlsson will compare it to the standard library
and he will explain how it's works with examples.
He will also details all the possibilities and the difference. For
examples in the first part of this book he explains, compares all the
pointer between them. So you can understand which pointer used in
which situation.


\section{Analyse}

This book is useful to have some basic knowledge about boost. You need
to have C++ skills before reading this book. but he will help you to
understand boost and to decide why you should use boost.
It's shows you the assets of boost.


@book{introductionboost,
title={Beyond the C++ Standard Library an Introduction to Boost},
author={Björn Karlsson},
  year={2006},
  publisher={Upper Saddle River (N.J.) : Addison-Wesley, cop. 2006},
	url={http://www.sudoc.fr/093832540}
}

\end{document}
