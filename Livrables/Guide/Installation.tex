\documentclass{article}
\usepackage[utf8]{inputenc}
\usepackage[T1]{fontenc}
\usepackage{hyperref}
\usepackage{verbatim} 
\title{Guide d'installation}
\author{Matéo Rémi}
\date{12/10/2012}
\begin{document}
\maketitle

\section{General installation for DGtal}

\subsection{Unix}
\begin{verbatim}
sudo apt-get install cmake g++ libboost-dev libboost-program-options-dev
sudo apt-get install libgmp-dev libcairo2-dev libqglviewer-qt4-dev
libgraphicsmagick++1-dev libinsighttoolkit3-dev libgdcm2-dev doxygen graphviz git
\end{verbatim}

after that you have to download the DGtal sources. You may need to register at this \href{https://github.com/}{website.}\\


\begin{verbatim}
git clone https://github.com/YOUR\_LOGIN\_GITHUB/DGtal.git
\end{verbatim}

after that you can cd on DGtal sources and create a temporary folder like tmp.
\begin{verbatim}
mkdir tmp
cd tmp
cmake ..
make
sudo make install
\end{verbatim}

ps : DO not install the lastest version about Boost because that is not compatible with DGtal

after that you can  make a dir like workspace where you will work.


create a file with the name CMakeLists.txt
and put this content.
\begin{verbatim}
PROJECT(Helloworld)
#Required in DGtal
CMAKE_MINIMUM_REQUIRED(VERSION 2.6)
FIND_PACKAGE(DGtal REQUIRED)
INCLUDE_DIRECTORIES(${DGTAL_INCLUDE_DIRS})
LINK_DIRECTORIES(${DGTAL_LIBRARY_DIRS})
ADD_EXECUTABLE(helloworld helloworld)
TARGET_LINK_LIBRARIES(helloworld ${DGTAL_LIBRARIES})
\end{verbatim}

when you are on unix link or copy the file DGtalConfig.cmake you can find it on /usr/local/share/DGtal

after that create files helloworld.cpp copy and paste the code below.

\begin{verbatim}
#include <DGtal/base/Common.h>
int main(int argc, char** argv)
{
DGtal::trace.info() << "Helloworld from DGtal ";
DGtal::trace.emphase() << "(version "<< DGTAL_VERSION << ")"<< std::endl;
return 0;
}
\end{verbatim}

and after that open terminal and go to the current directory and run cmake with the command:
\begin{verbatim}
cmake .
make
./helloworld
\end{verbatim}

\subsection{Windows}

On windows you can install it but you need a compiler so you have to choose between mingw or visual studio.

In the section getting started as I see it look like if you choose Visual that is more easy to configure.



\end{document} 
