\documentclass{article}
\usepackage[utf8]{inputenc}
\usepackage[T1]{fontenc}
\title{Rapport hebdomadaire}
\author{Matéo Rémi \& Plasse Vincent}
\date{12/10/2012}
\begin{document}
\maketitle

\section{Actions effectuées}
Téléchargement de la librairie digital tools et commencement de sa lecture. (2 heure Matéo Rémi) \\
Lecture de livres concernant Boost (2 heure Matéo Rémi) \\
Lecture de sources sur la géométrie discrète (3 h Vincent Plasse) \\
Réunion avec changement d'orientation du projet.  \\

\section{À faire} 
Mettre sur le rapport ce que l'on a expliqué pendant la réunion et lire des sources les concernant. (Vincent Plasse) \\
Améliorer la description de DGtal pour son utilisation mais aussi pour mieux faciliter sa compréhension.\\


\section{Notes}

Le projet se trouve être plus simple aprés cette réunion mais aussi beaucoup plus clair. \\
Nous avons compris commment le premier algorithme que nous allons implémenter marchait. \\
Il nous reste à trouver des sources et à mettre cela par écrit. \\

Nous pouvons également une fois que nous auront compris DGtal d'un point de vue technique à songer à la meilleur structure de données à adopter. \\

\end{document} 
