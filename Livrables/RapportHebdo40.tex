\documentclass{article}
\usepackage[utf8]{inputenc}
\usepackage[T1]{fontenc}
\title{Rapport hebdomadaire}
\author{Matéo Rémi \& Plasse Vincent}
\date{1-7/10/2012}
\begin{document}
\maketitle

\section{Actions effectuées}
Recherche d'articles et de sources sur Boost 1h \\
Redaction de la partie Thread (2h) \\
Telechargement et lecture du code source de DGtal (3h) \\
Lectures sur la géométrique discrète (3h) \\
 - Algorithmes de tranformation en distance \\
 - Représentation par axe médian \\
 - Transformations et reconstruction par projection\\


\section{À faire}
Trouver des sources bibliographiques plus précises sur la tranformation en distance \\
Trouver des articles traitant de l'utilisation de la librairie DGtal \\
Demander a la communauté utilisatrice leur recommandations de cette librairie \\

\section{Notes}

\subsection{Rémi Matéo}

Alors j'ai appris et je pense que nous avons assez de source pour avoir tout les renseignement dont nous pouvons avoir besoins sur Boost \\
DGtal est une librarie facile d'accés mais difficile à comprendre. Je pense donc chercher des articles utilisant DGtal afin de voir comment ils l'utilisent et pourquoi \\
La phase de modélisation sera complexe car le niveau de programmation de ces deux librairies est impressionnant\\

\subsection{Vincent Plasse}

Je commence à mieux discerner le fonctionnement des différents algorithmes de géométrie discrète et je discerne l'intérêt d'utiliser des algorithmes avec des nombres de parcours connus et restreints. Je n'ai malheureusement pas encore trouvé de source bibliographique précise sur la transformation en distance. C'est donc sur ce point que je vais accentuer mes recherches. \\

\end{document}
