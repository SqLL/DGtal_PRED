\documentclass{article}
\usepackage[utf8]{inputenc}
\usepackage[T1]{fontenc}
\title{Compte Rendu de Reunion}
\author{Matéo Rémi \& Plasse Vincent}
\date{Jeudi 13/09/2012}
\begin{document}
\maketitle

\section{Étaient présents}
Nicolas NORMAND \\
Eric REMY \\
Rémi MATEO \\
Vincent PLASSE \\


\section{Points abordées}
\subsection{Explication du PRED }
-	Fonctionnement par binômes, rendu final fin décembre et soutenance début janvier \\
-	3 étapes : bibliographie, modélisation et développement \\
-	Suivi par fiches hebdomadaires \\

\subsection{Contraintes ajoutées par N. NORMAND }
-	Livrables  en anglais \\
-	Utilisation de \LaTex et Markdown \\
-	Versioning sur le code et sur les livrables \\
Pour les fiches hebdomadaires, il est conseillé d’utiliser les logs du versioning pour préremplir la fiche \\
-	75% des sources bibliographiques doivent être pérennes \\

\subsection{Présentation de DGtal }
-	Bibliothèque de géométrie discrète \\
-	Développé par une communauté française \\
N. NORMAND a déjà implémenté un module pour exporter des objets géométriques en tikZ, il sera à utiliser pour le rapport \\
Rappel : \\
En Juin 2012 il y a eu une réunion DGtal dans laquelle E. REMY a proposé une extension : un algorithme de transformation en axe médian et un algorithme de transformation en distance. Le PRED concerne ce dernier. \\

\subsection{But }
Intégrer les algorithmes crées dans DGtal pour une plus grande visibilité et une utilisation simplifiée tout en pouvant effectuer des comparaisons avec d’autres algorithmes. \\

\subsection{Point sur DGtal }
-	Très forte utilisation de templates \\
-	Utilise la bibliothèque Boost \\
-	Permet d’implémenter des algorithmes de traitement indépendamment du format/de la dimension d’image en entrée \\
-	Utilise un système de concepts : on attache une signification (ex : est énumérable) à une ou plusieurs classes \\

\subsection{Intégration à DGtal }
Créer un outil qui utilisera la bibliothèque DGtal dans un premier temps afin de laisser les core développeurs choisir s’il y a lieu d’intégrer le code produit et de quelle façon \\
\\
Pour les choix structurels, il faudra le voir et les confirmer avec E.REMY qui propose d’ailleurs son code de test. \\

\subsection{A faire }
-	Lire la documentation de DGtal \\
-	Lire la liste d’ouvrages que N.NORMAND va fournir \\

\subsection{Point sur la communication }
-	Tout le monde reçoit tous les messages \\
-	L’objet des emails a pour préfixe [PRED DGtal] \\
-	Les prochaines réunions se feront avec Skype \\

\subsection{Introduction aux transformations en distance }
-	Rappel sur les notions de distance et de chemin \\
-	Distance euclidienne \\
-	Rappel sur les espaces discrets \\
-	Voisinage 4 et 8 connexe \\
-	Disque dans l’espace discret et problèmes \\
-	Nouvelles connexités et ajout de poids différents (ex : distance de chanfrein) \\
-	Combinaison de voisinages et de poids : distances basées sur des chemins \\
-	Présentation de l’algorithme parallèle \\
-	Présentation de l’algorithme de transformation en distance \\

      
\end{document}
